\section{Results} \label{sec:results}


\subsection{The Helium atom}
I will start presenting my main results: the ground state energies, and then 

\subsubsection{Ground state energies}
\begin{table} [H]
	\caption{}
	\begin{tabularx}{\textwidth}{X|X|X} \hline\hline
		\textbf{Method}&\textbf{Energy} [a.u.]&\textbf{Relative error}\\ \hline
		Experimental & -2.903694\footnote{Bergerson, 1998} & 0.00\%  \\
		Hylleraas & -2.90363 & 0.0022\% \\
		Reference & -2.750000 & 5.29\% \\
		CIS & -2.838648 &  2.24\% \\
		HF$_{\text{1 iteration}}$ & -2.829193 & 2.56\% \\
		HF$_{\text{converge}}$ & -2.831096 & 2.50\% \\ \hline\hline
	\end{tabularx}
\end{table}

\subsubsection{CIS}
Set up matrix and eigenvectors

\begin{equation}
\hat{H}_{\text{CIS}}=
\begin{pmatrix}
-2.750 & 0.179 & 0.179 & 0.088 & 0.088\\
0.179 & -2.080 & 0.044 & 0.101 & 0.022\\
0.179 & 0.044 & -2.080 & 0.022 & 0.101\\
0.088 & 0.101 & 0.022 & -2.023 & 0.012\\
0.088 & 0.022 & 0.101 & 0.012 & -2.023
\end{pmatrix}
\label{eq:H_He}
\end{equation}

\subsubsection{HF}
Set up matrix and eigenvectors after first iteration (basically when inserting C=I). 
Hartree-Fock matrix after one iteration (basically after inserting the identity matrix as the coefficients)
\begin{equation}
\hat{h}^{\text{HF}}=
\begin{pmatrix}
-0.750 & 0.000 & 0.179 & 0.000 & 0.088 & 0.000\\
0.000 & -0.750 & 0.000 & 0.179 & 0.000 & 0.088\\
0.179 & 0.000 & 0.296 & 0.000 & 0.180 & 0.000\\
0.000 & 0.179 & 0.000 & 0.296 & 0.000 & 0.180\\
0.088 & 0.000 & 0.180 & 0.000 & 0.164 & 0.000\\
0.000 & 0.088 & 0.000 & 0.180 & 0.000 & 0.164
\end{pmatrix}
\label{eq:h_HF_He_1iter}
\end{equation}
After the energy has converged, we find this matrix:
\begin{equation}
\hat{h}^{\text{HF}}=
\begin{pmatrix}
-0.840 & 0.000 & 0.226 & 0.000 & 0.102 & 0.000\\
0.000 & -0.840 & 0.000 & 0.226 & 0.000 & 0.102\\
0.226 & 0.000 & 0.271 & 0.000 & 0.169 & 0.000\\
0.000 & 0.226 & 0.000 & 0.271 & 0.000 & 0.169\\
0.102 & 0.000 & 0.169 & 0.000 & 0.159 & 0.000\\
0.000 & 0.102 & 0.000 & 0.169 & 0.000 & 0.159
\end{pmatrix}
\label{eq:h_HF_He_converge}
\end{equation}
Observe that they are not very different. Unlike the zeros in matrix \eqref{eq:h_HF_He_1iter}, the zeros in \eqref{eq:h_HF_He_converge} are not exactly zero, but smaller than 1e-17. 

\subsection{The Beryllium atom}

\begin{table} [H]
	\caption{}
	\begin{tabularx}{\textwidth}{X|X|X} \hline\hline
		\textbf{Method}&\textbf{Energy} [a.u.]&\textbf{Relative error}\\ \hline
		Experimental & -14.6674\footnote{Kramida, Martin, 1997} & 0.00\%  \\
		Reference & -13.7160 & 6.49\% \\
		CIS & -14.3621 &  2.08\% \\
		HF & -14.5083 & 1.08\% \\ \hline\hline
	\end{tabularx}
\end{table}

\iffalse
\begin{figure} [H]%
	\centering
	\subfloat[Lambda vs. R2-score]{{\includegraphics[width=7cm]{../plots/lambda_R2score.png} }}%
	\subfloat[Variance vs. R2-score]{{\includegraphics[width=7cm]{../plots/var_R2score.png} }}
	\caption{R$^2$-score plotted as a function of the penalty $\lambda$ (a) and as a function of the noise (b). $\lambda\in[10^{-8},10^2]$ in (a) and $\sigma^2\in[10^{-6},10^{-0.7}]$ in (b). The other parameters used were $\lambda=1e-5$ (penalty, was held constant for (b) only), $\eta=1e-4$ (learning rate), $niter=1e5$ (number of iterations) and $\mathcal{N}(0, \sigma^2=0.1)$ (noise, was held constant for (a) only).}%
	\label{fig:R2_scores}
\end{figure}
\fi 


