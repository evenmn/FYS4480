\section{Discussion} \label{sec:discussion}
The first thing that we can observe from table \eqref{tab:gs_he}, is how accurate Hylleraas' computation was. Creds to him! CIS turned out to be our best method, as expected since we used a basis of 5 SPFs. What is more surprising is that the HF energy was close, even though we only had one Slater Determinant. We also observe that CIS, HF$_{\text{1 iteration}}$ and HF$_{\text{converge}}$ all lie in between the reference energy and the experimental, which indicates that our calculations are correct. 

Switching to the CIS matrix for Helium, we see that it is symmetric as expected, and the diagonal elements are dominating. 

For the HF matrices, we observe that every second element is zero, which are the elements where the excited particle in the ket got opposite spin of the excited particle in the bra. We could find row equivalent matrices to the HF matrices with four block sections where two contained zeros only. We can also see that the two matrices are not very different, something that is reflected in the two HF energies.

For Beryllium, the HF energies are actually lower than the CIS energy, which is surprising. The reason could be that we only allow the electrons to be excited to one energy level above the Fermi level, with a more complex CIS matrix the energy would be lower for sure. The CIS and HF energies again lie in between the reference energy and the experimental energy, which is good. 