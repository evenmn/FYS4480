\section{Introduction} \label{sec:introduction}
The linear regression methods were first introduced for more than two centuries ago, and have been used in a large number of fields throughout the years [1][2]. In this project we will investigate whether the methods are sufficient for fitting polynomials to real terrain data, or we need more complicated methods. To challenge the methods, we chose terrain data from the volcanic island of Lombok, Indonesia, where the contour lines are quite dense. 

We developed our own software for ordinary least square (OLS), Ridge and Lasso linear regression, where the latter was based on minimization using gradient descent (GD). To verify the implementation, we tested it on data from the Franke function where we knew what the result should be. Further, the error was analyzed in order to decide which method that gave the best result, and all data was resampled using the K-fold validation method to estimate the actual error. 

For the results, see section \textit{Results} \eqref{sec:results}, which again is discussed in section \textit{Discussion} \eqref{sec:discussion}. The background theory can be found in section \textit{Theory} \eqref{sec:theory}, and all methods and techniques are presented in the section \textit{Methods} \eqref{sec:methods}. For code structure and implementation, see section \textit{Code} \eqref{sec:code}, and finally, the conclusion is found in section \eqref{sec:conclusion} with the same name.