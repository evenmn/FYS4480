\section{Introduction} \label{sec:introduction}
In the childhood of quantum mechanics, immense efforts were placed on Helium computations, because leading physicists found it crucial to provide calculations in agreement with experiments in order to prove the theory. The Norwegian physicist Egil Hylleraas calculated the ground state energy of the Helium atom with impressive accuracy already in 1929, which in some ways confirmed the validity of quantum mechanics. After that, physicists and chemists have managed to study systems of ever higher complexities, thanks to better methods and stronger computers.

In this project, we estimate the energy levels of the Helium and Beryllium atoms, with focus on the ground state energy. For doing that, we first apply the Configuration Interaction Singles (CIS), and thereafter turn to Hartree-Fock (HF). Can we compete with Hylleraas?

The background theory and notation is presented in section \eqref{sec:theory}, \textit{theory}, and the methods we used, in particular CIS and HF, are presented in section \eqref{sec:methods} \textit{methods}. We describe the code briefly in section \eqref{sec:code} \textit{code}, the results are in section \eqref{sec:results} with the same name, and the discussion and conclusion are given in section \eqref{sec:discussion} and \eqref{sec:conclusion} respectively. Finally, calculations of the matrix elements used in CIS are moved to Appendix A.
