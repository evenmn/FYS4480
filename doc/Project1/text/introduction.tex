\section{Introduction} \label{sec:introduction}
In the childhood of quantum mechanics, immense efforts were placed on Helium computations, because leading physicists found it crucial to provide calculations in agreement with experiments in order to prove the theory. The Norwegian physicist Egil Hylleraas calculated the ground state energy of the Helium atom with impressive accuracy already in 1929, which proved the quantum theory to be correct. After that, physicists and chemists have managed to study systems of ever higher complexities, thanks to better methods and stronger computers.

In this project we estimate the energy levels of the Helium and Beryllium atom, with focus on the ground state energy. For doing that, we first apply the Configuration Interaction Singles (CIS), and thereafter turn to Hartree-Fock. 


